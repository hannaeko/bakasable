\section{Think functions}

All the game logic is done by what we are going to call \textit{think functions}. A think function is a single function that is responsible of a certain task. That task should be atomic if possible.
Think function can be triggered by different ways, on action, loop or event. In this document those types are reffered as \textit{@on\_action}, \textit{@on\_loop}, \textit{@on\_event}.

The first one happens when a new action occures, either emitted by the local peer or a distant one. A \textit{@on\_action} think function take as arguments the type of action and the parametters of that action.

Next we have the \textit{@on\_loop} think function that will be executed at each loop iteration. To guarantee the atomicity of the think function it can be restrained to a specific entity. The function will then be called once for each object that is managed by the current peer with that object passed as argument.

At last, the \textit{@on\_event} think function is used as an handler for system event.
