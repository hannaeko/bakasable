\section{Names and data definition}
\subsection{Data structure}

\subsubsection{Base objects}

\paragraph{Primitives}
~
\begin{lstlisting}
Integer = <4 bytes signed integer>
UID64  = <8 bytes unsigned integer>
Character = <any utf-8 encoded unicode character>

String = *(Character) "0"

Number = Integer | UID64
Primitives = Numbers | String

GameId = UID64
PeerId = UID64
\end{lstlisting}

\paragraph{Compounds} Compounds structure agrated primitives together. The two main are \verb+Entity+ and \verb+ArrayOf+$\bigstar$.

~

The first is a group of primitives and is used as a base for every complex objects.

The second represent a general structure to create fixed-length array. It is construct by the length of the array followed by that amount of \verb+Entity+.

The \verb+OptionOf+$\bigstar$ structure at last permit the absence of value. A boolean will first indicate if a value follows followed by the optionnal value.

\begin{lstlisting}[escapechar=\%]
Entity = 1*(Primitives)

ArrayOf%$\bigstar$% := Length Length(%$\bigstar$%)
OptionOf%$\bigstar$% := "0" | "1" %$\bigstar$%
\end{lstlisting}

\subsubsection{Peers}

\begin{lstlisting}
Peer = PeerId String
ArrayOfPeer
\end{lstlisting}


\subsubsection{Game}
The base types of game object are \verb+GameObject+ one of those is composed of
the \verb+GameObjectId+, that is the unique id of the type, followed by the
definition of the \verb+Entity+.

~

We count currently three types of \verb+GameObject+, the \verb+MapChunk+, the
\verb+Chunk+ and the \verb+Sheep+.

~

The \verb+MapChunk+ represent a section of terrain of 15 blocks by 15 blocks. The \verb+MapChunk+ is composed of its unique id, coordinates and blocks data. In opposition to most of UID, the \verb+MapChunk+ UID is not random as it should be predictable to allow peers to load start a fetch process for the chunk.

Another particularity of that \verb+GameObject+ is the coordinate system. The \verb+MapChunk+ is index over \textit{chunk coordinated} that are equal to the world coordinate divided by the length of a chunk edge (15). Therefore the chunk $(1,1)$ represent the chunk that goes from $(15, 15)$ to $(29,29)$ (include).

~

The \verb+Chunk+ is used to aggregate a \verb+MapChunk+ with the entities present on this chunk. It has no real use other than being a container for transfert over the network. It still has a UID to keep the consistency among \verb+GameObject+s but as it is currently only used for debugging that id can be safely set to the id of the \verb+ChunkMap+.

\begin{lstlisting}
GameObjectId = Integer
GameObject = GameObjectId Entity

MapChunk = "1" UID64 Integer Integer ArrayOfArrayOfNumber
Chunk = "2" UID64 MapChunk ArrayOfGameObject
Sheep = "3" UID64 Interger Interger String
\end{lstlisting}

\subsubsection{Network}

Network data structure are only used to transfer information over the network. It can embed other data structure or just an action.

~

The \verb+Result+ structure is used when retrieving objects via the fetch process. The first component is the status code followed by an optional \verb+GameObject+ if the result is successful.

~

The list of status code is the following:
\begin{itemize}
  \item 100, \verb+OK+, the request is successful, the object has been fetched and follows;
  \item 201, \verb+DELETED_ENTITY+, the requested object could not be retrieved because it has been deleted.
\end{itemize}

\begin{lstlisting}
Result = Status OptionOfGameObject
Status = Integer
\end{lstlisting}

\subsection{Interest names}

\begin{lstlisting}
Broadcast := <"ndn"><"broadcast"><"mygame"><GameId>
LocalRoute := <>+<"mygame"><GameId><PeerId>

JoinInterest := <Broadcast><"join"><LocalRoute>
LeaveInterest := <Broadcast><"leave"><PeerId>

ChunkEntitiesInterest := <LocalRoute><"chunk"><Integer><Integer><"entities">
FetchEntityInterest := <LocalRoute><"entity"><UID64><"fetch">

FindCoordinatorInterest := <Broadcast><"coordinator"><UID64><PeerId>
FindEntityInterest := <Broadcast><"entity"><UID64><PeerId>
EntityFoundInterest := <LocalRoute><"entity_found"><UID64><PeerId>
\end{lstlisting}
